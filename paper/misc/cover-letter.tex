% Template for a letter on IAG-USP letterhead

%%%%%%%%%%%%%%%%%%%%%%%%%%%%%%%%%%%%%%%%%%%%%%%%%%%%%%%%%%%%%%%%%%%%%%%%%%%%%%%
% Set a class and general configuration
\documentclass[a4paper,onecolumn,10pt]{article}

%%%%%%%%%%%%%%%%%%%%%%%%%%%%%%%%%%%%%%%%%%%%%%%%%%%%%%%%%%%%%%%%%%%%%%%%%%%%%%%
% Set variables with the title, authors, etc.

% Used to set information about the paper that is used in multiple files
%%%%%%%%%%%%%%%%%%%%%%%%%%%%%%%%%%%%%%%%%%%%%%%%%%%%%%%%%%%%%%%%%%%%%%%%%%%%%%%

% Set variables with the title, authors, etc.
\newcommand{\Title}{Euler inversion: Locating sources of potential-field data through inversion of Euler's homogeneity equation}
\newcommand{\TitleShort}{Euler inversion}

\newcommand{\Year}{2024}
\newcommand{\PreprintOn}{2024/12/19}
\newcommand{\SubmittedOn}{2024/12/20}
\newcommand{\RevisionAOn}{2025/02/28}
% \newcommand{\PublishedOn}{2023/02/28}

\newcommand{\AuthorShort}{Uieda et al.}
\newcommand{\Authors}{%
  Leonardo Uieda\textsuperscript{1},
  Gelson Ferreira Souza-Junior\textsuperscript{1},
  India Uppal\textsuperscript{2},
  Vanderlei Coelho Oliveira Jr.\textsuperscript{3}
}
\newcommand{\Email}{uieda@usp.br}
\newcommand{\Corresponding}{%
  Corresponding author: Leonardo Uieda <\href{mailto:\Email}{\Email}>
}
\newcommand{\Affiliations}{%
  \textsuperscript{1} Universidade de São Paulo, Brazil;
  \textsuperscript{2} University of Liverpool, UK;
  \textsuperscript{3} Observatório Nacional, Brazil;
}
\newcommand{\AuthorORCIDs}{%
  LU (\href{https://orcid.org/0000-0001-6123-9515}{0000-0001-6123-9515});
  GFSJ (\href{https://orcid.org/0000-0002-5695-4239}{0000-0002-5695-4239});
  IU (\href{https://orcid.org/0000-0003-3531-2656}{0000-0003-3531-2656});
  VCOJr (\href{https://orcid.org/0000-0002-6338-4086}{0000-0002-6338-4086})
}

\newcommand{\Journal}{Geophysical Journal International}
\newcommand{\JournalDOI}{YYYYY/YYYYYYY}
\newcommand{\PreprintDOI}{10.31223/X5T41M}
\newcommand{\ArchiveDOI}{10.6084/m9.figshare.26384140}
\newcommand{\GitHubRepository}{compgeolab/euler-inversion}

% From GJI's list of keywords: https://static.primary.prod.gcms.the-infra.com/static/site/gji/document/GJI%20Keywords%202023.pdf?node=ef215d8b2a32e1ee1ffe
\newcommand{\Keywords}{%
  Gravity anomalies and Earth structure;
  Magnetic anomalies: modelling and interpretation;
  Inverse theory;
}

\newcommand{\LetterTitle}{Cover letter -- \Title}

%%%%%%%%%%%%%%%%%%%%%%%%%%%%%%%%%%%%%%%%%%%%%%%%%%%%%%%%%%%%%%%%%%%%%%%%%%%%%%%
% Import the required packages
\usepackage[utf8]{inputenc}
\usepackage[TU]{fontenc}
\usepackage[british]{babel}
\usepackage{graphicx}
\usepackage{hyperref}
\usepackage{fancyhdr}
\usepackage{geometry}
\usepackage{microtype}
\usepackage{xcolor}
% improved urls with proper hyphenation
\usepackage{xurl}
% Use a different font
\usepackage[scaled=1.1]{notomath}
% Control the font size
\usepackage{anyfontsize}
\usepackage{setspace}
% Generate random text
\usepackage{lipsum}
% Better left and right align
\usepackage{ragged2e}

%%%%%%%%%%%%%%%%%%%%%%%%%%%%%%%%%%%%%%%%%%%%%%%%%%%%%%%%%%%%%%%%%%%%%%%%%%%%%%%
% Configuration of the document

\geometry{%
  left=25mm,
  right=15mm,
  top=15mm,
  bottom=15mm,
  headsep=15mm,
  headheight=15mm,
  footskip=7mm,
  includehead=true,
  includefoot=true
}

% Control line spacing
\onehalfspacing
\newcommand{\Padding}{\vspace{0.5cm}}

% Custom colors
\definecolor{darkgray}{gray}{0.4}
\definecolor{mediumgray}{gray}{0.5}
\definecolor{lightgray}{gray}{0.9}
\definecolor{mediumblue}{HTML}{2060c2}
\definecolor{lightblue}{HTML}{f7faff}

% Make urls use the same font as every other text
\urlstyle{same}

% Configure hyperref and add PDF metadata
\hypersetup{
    colorlinks,
    allcolors=mediumblue,
    pdftitle={\LetterTitle},
    pdfauthor={\AuthorShort},
    breaklinks=true,
}

%%%%%%%%%%%%%%%%%%%%%%%%%%%%%%%%%%%%%%%%%%%%%%%%%%%%%%%%%%%%%%%%%%%%%%%%%%%%%%%
\begin{document}

\begin{flushleft}
    Editorial team
    \\
    \Journal
\end{flushleft}
\begin{flushright}
  \today
\end{flushright}
\Padding

\noindent
Dear Editor,
\Padding

We are pleased to submit the original research paper titled ``\Title{}'' for
consideration by \Journal{}.

The paper introduces a novel method that we have developed, which we call
\textit{Euler inversion}, for estimating the location and approximate geometry
of sources of gravity and magnetic anomalies.
Euler inversion is a significant departure from established methods (mainly
Euler deconvolution and its variants), both in terms of the mathematical
formulation and the higher robustness of the results to random noise and the
presence of interfering sources.
Instead of a linear inversion, we perform a non-linear inversion which can
estimate both the model parameters and the predicted potential-field and its
spatial derivatives.
The mathematics used are similar to total-least-squares from statistics and the
combined adjustment from geodesy (which was the inspiration for our work).
An added benefit of being able to provide predicted data (which Euler
deconvolution cannot do) is that Euler inversion can also estimate an integer
structural index, which is related to the source geometry.
We found that the best index is the one that minimizes the data misfit.
Euler inversion is also fast to execute on a standard computer, making it
a practical alternative to Euler deconvolution on an Earth scientists toolbox.
All of the code and data needed to reproduce every figure is provided through
the DOI and GitHub repository to which there are links in the paper.

Thank you for your consideration of our manuscript.
\Padding

\begin{flushleft}
  Sincerely,
  \Padding

  India Uppal (on behalf of all authors)
  \\[0.25cm]
  {
    \color{mediumgray}
    \small
    Department of Earth, Ocean and Ecological Sciences
    \\
    School of Environmental Science
    \\
    University of Liverpool
    \\
    United Kingdom
  }
\end{flushleft}
\end{document}
