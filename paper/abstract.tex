Locating the sources of observed disturbances in potential-field data is a challenging problem due to the non-unique nature of the inverse problem.
The Euler deconvolution method was created to solve this issue, particularly for idealized sources (such as spheres, planar vertical dykes).
Euler deconvolution has become widely used in potential-field methods due, in large part, to its low computational cost and ease of implementation into software.
However, it is widely known that Euler deconvolution suffers from some shortcomings: 1) non-uniqueness of the solution with respect to the depth of the source and the structural index (a parameter that represents the idealised shape of the source); 2) sensitivity to short-wavelength noise in the data derivatives which are used as inputs for the method.
Here, we present a new method called \textit{Euler inversion} which is a reformulation of the inverse problem of Euler's homogeneity equation as an implicit mathematical model rather than a parametric one.
Euler inversion is a constrained, non-linear inverse problem capable of estimating both the model parameters (location of the source and constant base level) and the predicted data (potential field and its derivatives).
We show that Euler inversion is less sensitive than Euler deconvolution to short-wavelength noise and to the presence of interfering sources in the data window.
By also estimating the predicted data, Euler inversion is also able to estimate the integer structural index because the data misfit is smaller when the correct structural index is used for the inversion.
Furthermore, most matrices involved in the method are either sparse or diagonal, making Euler inversion computationally efficient.
Tests on synthetic data and a real aeromagnetic dataset from Rio de Janeiro, Brazil, demonstrate the effectiveness of Euler inversion to delineate sources with variable geometries and correctly estimate their depths.
