Earth scientists are able to determine the depth of certain rocks in below ground by measuring the small disturbances that they cause to Earth's gravity and magnetic fields.
A popular method for doing this is \textit{Euler deconvolution}, which is widely available in geoscience software and is fast to run on an average computer. 
Unfortunately, Euler deconvolution has some shortcomings: 1) the approximate shape of the rocks must be known, for example a sphere or a wide flat slab, which is represented by something called the \textit{structural index} 2) the depth of the rocks is not well estimated when there is noise in our data, which is a common occurrence.
We propose a new method that uses more adequate (and complex) mathematics to fix some of the shortcomings of Euler deconvolution.
We call our new method \textit{Euler inversion}.
It is less sensitive to noise in the data and is also able to determine the approximate shape of the source (the structural index).
Our new method is able to replace Euler deconvolution on an Earth scientists toolbox because it is also fast to execute on an average computer.