Earth scientists can estimate the depth of certain rocks beneath Earth's surface by measuring the small disturbances that they cause in the Earth's gravity and magnetic fields.
A popular method for this is \textit{Euler deconvolution}, which is widely available in geoscience software and can be run quickly on a standard computer. 
Unfortunately, Euler deconvolution has some shortcomings: 1) the approximate shape of the rocks must be known, for example, a sphere or a wide flat slab, represented by the \textit{structural index} 2) the depth of the rocks is not well estimated when there is noise in our data, which is a common occurrence.
We propose a new method, \textit{Euler inversion}, which fixes some of the shortcomings of Euler deconvolution by using more adequate (and complex) mathematics.
Our method is less sensitive to noise in the data and is also able to determine the approximate shape of the source (the structural index).
Euler inversion is also fast to execute on a standard computer, making it a practical alternative to Euler deconvolution on an Earth scientists toolbox.