\documentclass[onecolumn]{article}

\usepackage{amsmath}
\usepackage{graphicx}
\usepackage{url}
\usepackage[round]{natbib}
\usepackage[utf8]{inputenc}
% To make two column figures appear in the correct order
\usepackage{fixltx2e}
% To make annotations on the sides
\usepackage{todonotes}
% To generate dummy text
\usepackage{lipsum}
% Metadata for the generated PDF
\usepackage[pdftex,colorlinks=true]{hyperref}
\hypersetup{
    allcolors=blue,
}


\begin{document}

\title{Implicit models in geophysical inverse problems}
\author{
    Leonardo Uieda$^{1}$
    and
    Vanderlei C. Oliveira Jr.$^{2}$
    \\\\
    {\small $^1$Universidade do Estado do Rio de Janeiro, Rio de Janeiro, Brazil.
        e-mail: leouieda@gmail.com}
    \\
    {\small $^2$Observatório Nacional, Rio de Janeiro, Brazil.
        e-mail: vandscoelho@gmail.com}
}


\maketitle


\begin{abstract}
    \lipsum[1]
\end{abstract}


%%%%%%%%%%%%%%%%%%%%%%%%%%%%%%%%%%%%%%%%%%%%%%%%%%%%%%%%%%%%%%%%%%%%%%%%%%%%%%%
\section{Introduction}

Cite things using \citet{tikhonov1977} or \citep{tikhonov1977}.

%\begin{figure}
    %\centering
    %\includegraphics[]{figures/example}
    %\caption{
        %Example image.
    %}
    %\label{fig:meh}
%\end{figure}


%%%%%%%%%%%%%%%%%%%%%%%%%%%%%%%%%%%%%%%%%%%%%%%%%%%%%%%%%%%%%%%%%%%%%%%%%%%%%%%
\section{Classical theory of implicit models (old stuff)}

Show the basic derivation of the combined adjustment in Krakiwsky's lecture
notes:
\url{http://www2.unb.ca/gge/Pubs/LN42.pdf} and
\url{http://www2.unb.ca/gge/Pubs/LN18.pdf}.

Following \citet{vanicek1986}.

Derive equations for the combined adjustment with and without $\delta^T\delta$.

Note that the observed data $\mathbf{l}$
are only included
as the initial estimate
for the predicted data $\mathbf{d}^0$.
So after the first iteration
the observed data are not present
in any of the equations.

Definitions:
$\mathbf{d}$ is the $N$-dimensional vector of predicted data,
$\mathbf{l}$ is the $N$-dimensional vector of observed data,
$\mathbf{p}$ is the $M$-dimensional vector of model parameters,

In traditional geophysical inversion theory,
we assume that there is a functional relation between
the model parameters $\mathbf{p}$
and the predicted data $\mathbf{d}$

\begin{equation}
    \mathbf{d} = \mathbf{f}(\mathbf{p}),
    \label{eq:parametric}
\end{equation}

\noindent
in which $\mathbf{f}$ is a vector representing the $L = N$ functions
$f_1, f_2, \ldots, f_L$.
Equation~\ref{eq:parametric} represents a parametric model.

\todo{Write the whole solution? (Misfit + Gauss-Newton solution)}


Sometimes it is impossible to separate the data from the model parameters.
In such cases,
we must use an implicit mathematical model

\begin{equation}
    \mathbf{f}(\mathbf{p}, \mathbf{d}) = \mathbf{0}.
    \label{eq:implicit}
\end{equation}

$\mathbf{f}$ is a vector of $L$ equations.

We must estimate both the model parameters and the predicted data.

First, we linearize Equation~\ref{eq:implicit} by expanding it to first order
in a Taylor series around $\mathbf{p}^0$ and $\mathbf{d}^0$

\begin{equation}
    \mathbf{f}^0 + \mathbf{A}\mathbf{\delta} + \mathbf{B}\mathbf{v} =
    \mathbf{0},
    \label{eq:implicit-taylor}
\end{equation}

\noindent
in which
$\mathbf{f}^0 = \mathbf{f}(\mathbf{p}^0, \mathbf{d}^0)$,
$\mathbf{\delta} = \mathbf{p} - \mathbf{p}^0$,
$\mathbf{v} = \mathbf{d} - \mathbf{d}^0$,
$\mathbf{A}$ is
the $L \times M$ dimensional Jacobian matrix of $\mathbf{f}$
with respect to $\mathbf{p}$,
and
$\mathbf{B}$ is
the $L \times N$ dimensional Jacobian matrix of $\mathbf{f}$
with respect to $\mathbf{d}$.
The element $A_{ij}$ of matrix $\mathbf{A}$ is

\begin{equation}
    A_{ij} = \dfrac{\partial f_i}{\partial p_j},
    \label{eq:A}
\end{equation}

\noindent
and the element $B_{ij}$ of matrix $\mathbf{B}$ is

\begin{equation}
    B_{ij} = \dfrac{\partial f_i}{\partial d_j}.
    \label{eq:B}
\end{equation}

%%%%%%%%%%%%%%%%%%%%%%%%%%%%%%%%%%%%%%%%%%%%%%%%%%%%%%%%%%%%%%%%%%%%%%%%%%%%%%%
\section{Adaptations (need better name for this) (new stuff)}

Formulate the adjustment using $r^Tr$.
Show that the Gauss-Newton method
for non-linear problems
is a particular case of $B = -I$.
Show that the linear problem
is a particular case of $A$ not depending on $p$
and $p^0 = 0$.

Formulate the adjustment with $\delta^T\delta$
and show that the method of Marquardt
is a particular case of this.



\subsection{Regularization}

Add Tikhonov regularization to the combined adjustment derived above.



\subsection{Calculating synthetic data}

Show how to use Newton's root finding method to calculate predicted data from a
given $p$. Only valid when $L = N$.

Show least squares solution for under- and over-determined problems using
$f^tf$.

%%%%%%%%%%%%%%%%%%%%%%%%%%%%%%%%%%%%%%%%%%%%%%%%%%%%%%%%%%%%%%%%%%%%%%%%%%%%%%%
\section{Implicit models in geophysics}

Give examples here of implicit models.
If only one example


\subsubsection{Heat flow  at the base of the crust}

\subsubsection{Love wave dispersion}


%%%%%%%%%%%%%%%%%%%%%%%%%%%%%%%%%%%%%%%%%%%%%%%%%%%%%%%%%%%%%%%%%%%%%%%%%%%%%%%
\section{Acknowledgments}

We are indebted to the developers and maintainers of the open-source
software without which this work would not have been possible.

%%%%%%%%%%%%%%%%%%%%%%%%%%%%%%%%%%%%%%%%%%%%%%%%%%%%%%%%%%%%%%%%%%%%%%%%%%%%%%%

\bibliographystyle{plainnat}
\bibliography{references}

\end{document}
