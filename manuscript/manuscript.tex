\documentclass[twocolumn]{article}

\usepackage{amsmath}
\usepackage{graphicx}
\usepackage{url}
\usepackage[round]{natbib}
\usepackage[utf8]{inputenc}
% To make two column figures appear in the correct order
\usepackage{fixltx2e}
% To make annotations on the sides
\usepackage{todonotes}
% To generate dummy text
\usepackage{lipsum}
% Metadata for the generated PDF
\usepackage[pdftex,colorlinks=true]{hyperref}
\hypersetup{
    allcolors=blue,
}


\begin{document}

\title{
    Implicit models in geophysical inverse problems
}
\author{
    Leonardo Uieda$^{1}$
    and
    Vanderlei C. Oliveira Jr.$^{2}$
    \\\\
    {\small
        $^1$Universidade do Estado do Rio de Janeiro, Rio de Janeiro, Brazil.
        e-mail: leouieda@gmail.com
    }
    \\
    {\small
        $^2$Observatório Nacional, Rio de Janeiro, Brazil.
    }
}


\maketitle


\begin{abstract}
    \lipsum[1]
\end{abstract}


%%%%%%%%%%%%%%%%%%%%%%%%%%%%%%%%%%%%%%%%%%%%%%%%%%%%%%%%%%%%%%%%%%%%%%%%%%%%%%%
\section{Introduction}

Cite things using \citet{tikhonov1977} or \citep{tikhonov1977}.

%\begin{figure}
    %\centering
    %\includegraphics[]{figures/example}
    %\caption{
        %Example image.
    %}
    %\label{fig:meh}
%\end{figure}


%%%%%%%%%%%%%%%%%%%%%%%%%%%%%%%%%%%%%%%%%%%%%%%%%%%%%%%%%%%%%%%%%%%%%%%%%%%%%%%
\section{Classical theory of implicit models (old stuff)}

Show the basic derivation of the combined adjustment in Krakiwsky's lecture
notes:
\url{http://www2.unb.ca/gge/Pubs/LN42.pdf} and
\url{http://www2.unb.ca/gge/Pubs/LN18.pdf}.

Derive equations for the combined adjustment with and without $\delta^T\delta$.

Note that the observed data $\mathbf{l}$
are only included
as the initial estimate
for the predicted data $\mathbf{d}^0$.
So after the first iteration
the observed data are not present
in any of the equations.



%%%%%%%%%%%%%%%%%%%%%%%%%%%%%%%%%%%%%%%%%%%%%%%%%%%%%%%%%%%%%%%%%%%%%%%%%%%%%%%
\section{Adaptations (need better name for this) (new stuff)}

Formulate the adjustment using $r^Tr$.
Show that the Gauss-Newton method
for non-linear problems
is a particular case of $B = -I$.
Show that the linear problem
is a particular case of $A$ not depending on $p$
and $p^0 = 0$.

Formulate the adjustment with $\delta^T\delta$
and show that the method of Marquardt
is a particular case of this.



\subsection{Regularization}

Add Tikhonov regularization to the combined adjustment derived above.



\subsection{Calculating synthetic data}

Show how to use Newton's root finding method to calculate predicted data from a
given $p$. Only valid when $L = N$.

Show least squares solution for under- and over-determined problems using
$f^tf$.

%%%%%%%%%%%%%%%%%%%%%%%%%%%%%%%%%%%%%%%%%%%%%%%%%%%%%%%%%%%%%%%%%%%%%%%%%%%%%%%
\section{Implicit models in geophysics}

Give examples here of implicit models.
If only one example


\subsubsection{Heat flow  at the base of the crust}

\subsubsection{Love wave dispersion}


%%%%%%%%%%%%%%%%%%%%%%%%%%%%%%%%%%%%%%%%%%%%%%%%%%%%%%%%%%%%%%%%%%%%%%%%%%%%%%%
\section{Acknowledgments}

We are indebted to the developers and maintainers of the open-source
software without which this work would not have been possible.

%%%%%%%%%%%%%%%%%%%%%%%%%%%%%%%%%%%%%%%%%%%%%%%%%%%%%%%%%%%%%%%%%%%%%%%%%%%%%%%

\bibliographystyle{plainnat}
\bibliography{references}

\end{document}
